\documentclass[12pt]{myarticle}

\usepackage{enumitem}
\usepackage{listings}
\usepackage{color} %red, green, blue, yellow, cyan, magenta, black, white
\definecolor{mygreen}{RGB}{28,172,0} % color values Red, Green, Blue
\definecolor{mylilas}{RGB}{170,55,241}
\usepackage{float}
\usepackage{hyperref}
\begin{document}

\lstset{language=Matlab,%
    %basicstyle=\color{red},
    breaklines=true,%
    morekeywords={matlab2tikz},
    keywordstyle=\color{blue},%
    morekeywords=[2]{1}, keywordstyle=[2]{\color{black}},
    identifierstyle=\color{black},%
    stringstyle=\color{mylilas},
    commentstyle=\color{mygreen},%
    showstringspaces=false,%without this there will be a symbol in the places where there is a space
    numbers=left,%
    numberstyle={\tiny \color{black}},% size of the numbers
    numbersep=9pt, % this defines how far the numbers are from the text
    emph=[1]{for,end,break},emphstyle=[1]\color{red}, %some words to emphasise
    %emph=[2]{word1,word2}, emphstyle=[2]{style},    
}

\section*{Problem Set 4}
\subsection*{4.5 (b) Data Set}
Firstly, I choose the hyperparameter ($\lambda$ \& $\pi$) for different cluster number K.

When K = 1, we set hyperparameter $\pi$ = 1, $\lambda$ = 1.

\begin{table}[H]
    \centering
    \begin{tabular}{|c|c|c|}
    \hline
    K = 1       & $\pi$ & $\lambda$  \\ \hline
    London & 1  & 0.92882 \\ \hline
    Antwerp & 1  & 0.89583 \\ \hline
    \end{tabular}
\end{table}

When K = 2, we set hyperparameter $\pi$ = [0.5,0.5], $\lambda$ =[1,2].

\begin{table}[H]
    \centering
    \begin{tabular}{|c|c|c|}
    \hline
    K = 1       & $\pi$  & $\lambda$  \\ \hline
    London & [0.57883,0.42117]  & [0.86540,1.01598] \\ \hline
    Antwerp & [0.66110,0.33890]  & [0.22974,2.19520] \\ \hline
    \end{tabular}
\end{table}

When K = 3, we set hyperparameter $\pi$ = [0.33,0.33,0.34], $\lambda$ =[1,2,3].

\begin{table}[H]
    \centering
    \begin{tabular}{|c|c|c|}
    \hline
    K = 1       & $\pi$  & $\lambda$  \\ \hline
    London & [0.475,0.326,0.199]  & [0.835,1.006,1.028] \\ \hline
    Antwerp & [0.401,0.314, 0.285]  & [0.089,0.613,2.344] \\ \hline
    \end{tabular}
\end{table}

When K = 4, we set hyperparameter $\pi$ = [0.25,0.25,0.25,0.25], $\lambda$ =[1,2,3,4].

\begin{table}[H]
    \centering
    \begin{tabular}{|c|c|c|}
    \hline
    K = 1       & $\pi$  & $\lambda$  \\ \hline
    London & [0.441,0.295,0.171,0.093]  & [0.827,0.997,1.020,1.027] \\ \hline
    Antwerp & [0.412,0.302,0.158,0.128]  & [0.096,0.619,2.339,2.339] \\ \hline
    \end{tabular}
\end{table}

When K = 5, we set hyperparameter $\pi$ = [0.2,0.2,0.2,0.2, 0.2], $\lambda$ =[1,2,3,4,5].

\begin{table}[H]
    \centering
    \begin{tabular}{|c|c|c|}
    \hline
    K = 1       & $\pi$  & $\lambda$  \\ \hline
    London & [0.426,0.282,0.161,0.087,0.045]  & [0.824,0.992,1.016,1.023,1.027] \\ \hline
    Antwerp & [0.420,0.293,0.122,0.098,0.067]  & [0.101,0.623,2.336,2.336,2.336] \\ \hline
    \end{tabular}
\end{table}

\subsection*{Conclusion}
Observe the experiment results, we can find that:

\begin{enumerate}[label=(\roman*)]
    \item For London, the cells could be approximately divided into two clusters: 45\%($\pi$) 0.8($\lambda$); 55\%($\pi$) 1.0($\lambda$).
    \item For Antwerp, the cells could be approximately divided into two clusters: 40\%($\pi$) 0.85($\lambda$); 60\%($\pi$) 2.3($\lambda$).
\end{enumerate}

Here, $\lambda$ denotes the mean of number of hits. Therefore, compared to London, Antwerp has a large space that gets more hits. The hit frequency of areas in Antwerp has big difference. Enemies have some target areas in Antwerp. Citizens in Antwerp should avoid staying in this dangerous region. The hit frequency of areas in London tends to be the same (0.8 vs 1.0). It seems that enemies have no target areas in London. Citizens in London should also avoid staying in the more dangerous region.

\subsection*{Codes}

Source code can be found at \url{https://github.com/yangji12138/machine-learning/tree/master/PS4}.

\lstinputlisting{../costFunction.m}

\lstinputlisting{../computeGamma.m}

\lstinputlisting{../computelambda.m}

\lstinputlisting{../computePi.m}

\lstinputlisting{../EM.m}


\end{document}